The generated point cloud was saved as a text file in the \texttt{UOS\_RGB} file format. The first line represented the number of 3D object surface points and the following lines had $(x, y, z)$ coordinates of each point respectively with their $(R, G, B)$ information defined in a left-handed coordinate system. 
Each complete scan of the target object was in this way stored as a new \texttt{scanXXX.3d} file where \texttt{XXX} determines the index of the scan. 

We used \ac{ICP} \cite{besl:1992} to register the two points from different scans into a common coordinate system using equation \ref{equation:icp}. The algorithm requires an initial starting guess of relative poses $(x, y, z, \theta_x, \theta_y, \theta_z)$ to compute the rotation and translation to fit the 3D geometrical models together. Since our system did not have an odometer, we set the initial pose to 0 and let it extrapolate further. 

\begin{align}
	\label{equation:icp}				
	E(R, t) &= \sum_{i=1}^{N_m} \sum_{j=1}^{N_d} w_{i,j}\| m_i - (Rd_j + t) \|^2 \notag \\
	\text{where}~ N_m &= \text{number of points in model set}~ M \notag \\
						N_d &= \text{number of points in data set}~ D \notag \\
						w_{i,j} &= 1, ~\text{if}~ m_i~ \text{is closest to}~	d_j \notag \\
						w_{i,j} &= 0, ~\text{otherwise}
\end{align}

We used 6D \ac{SLAM} from \ac{3DTK} \cite{3dtk:2012} for automatic scan registration that uses cached kd-trees for fast iterative \ac{ICP} match \cite{nuchter:2007}. The \texttt{slam6D} component produced a \texttt{frames} file that was used by the \texttt{show} component to visualize the 3D model of the target object along with its color information.