Stereophotogrammetry \cite{li:2004, debevec:1996} is a technique to generate
3D models using a collection of complementary 2D images of an object taken at
different angles It uses perspective methods and illumination rules to recover
basic geometric models of the photographed scenes. The hardware requirements
of the setup are usually two digital cameras, however the quality of the
digitized 3D models is generally low.

Structured Light is an active contact-free technique where a known pattern is
projected onto the surface and a separate perspective is used to observe the
resulting deformed pattern. The 2D deformed image of the pattern acquired is
used to extract the 3D information of the surface. Microsoft Kinect
\footnote{\href{http://www.xbox.com/en-us/kinect}{http://www.xbox.com/en-us/kinect}}
internally this technique to compute a depth of the object by projecting an
infrared laser. The major limitation of the Kinect is its inability to work
well in outdoor environments.

The time-of-flight scanning approach \cite{nielsen:1996} is also
an active contact-free technique where the camera measures the round trip time
of the light signal from the object surface to determine its distance. Such an
approach can capture depth information over a long-range and are fairly quick
in operation. Unfortunately the equipments are expensive and the accuracy is
not high.

Contact-free triangulation based techniques to generate 3D models of
real-world objects have been known for more than decade \cite{beraldin:1999,
cortelazzo:2004}. The traditional systems however, use high-precision
expensive actuators for rotating/translating the laser plane and the object.
They also depend on external sensors to track the position of the scanner.

The underpinnings of our cataloging system are largely inspired from The David
Laser Scanner \cite{winkelbach:2006} project.  It is a software package that
uses the triangulation technique, however with a self-calibration method for
the hand-held laser plane to keep the cost to a minimum. It is a
generalization of \cite{zagorchev:2006} which instead uses four visual
intersection points of laser with a double reference frame to calibrate the
laser.The software package although originally free, is now available at a
price of \euro 299. In addition, it can only run on Windows since currently it
is dependent on the .NET framework. We in this paper present a free
alternative to the David Laser Scanner. It is written in C++ and is based on
OpenCV and the "3DTK - The 3D Toolkit" \cite{3dtk:2012} making it
widely-available as a cross-platform solution.
